O número de sequências biológicas disponíveis aumentou significativamente nos últimos anos devido a várias descobertas científicas sobre o código genético que compõe os seres vivos, criando um enorme volume de dados. Por consequência, novos métodos computacionais foram moldados para analisar e extrair informações dessas sequências genéticas. Os métodos de aprendizagem de máquina têm mostrado ampla aplicabilidade em bioinformática e demonstrou ser imprescendível para a extração de informações úteis das estruturas secundárias dos genomas ao aperfeiçoar suas técnicas com base no arquétipo matemático em contraste com o modelo padrão biológico de análise. No entanto, ainda existem vários problemas que motivam novas abordagens, principalmente envolvendo problemas de extração de características em estruturas extremamente pequenas como os snoRNAs. Considerando isso, o foco do trabalho é estudar e analisar os algoritmos de extração de características baseado em modelos matemáticos como o mapeamento numérico, a transformação de Fourier, entropia e redes complexas, tendo como estudo de caso as duas classes principais de snoRNAs: \textit{C/D box} e \textit{H/ACA box}. De forma concisa, definimos o estudo de caso em um pipeline dividido em ciclos fundamentado nas bases do \textit{machine learning} que guiará a pesquisa solidificando o comparativo entre o paradigma matemático adotado e os métodos biológicos casuais. 

\begin{keywords}
RNAs não codificadores, snoRNAs, Aprendizagem de Máquina, Modelos matemáticos de extração de características
\end{keywords}