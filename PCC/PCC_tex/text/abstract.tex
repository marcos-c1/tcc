The number of available biological sequences has increased significantly in recent years due to several scientific discoveries about the genetic code stored in living organisms, creating a huge volume of data. Consequently, new computational methods were shaped to analyze and extract information from these genetic sequences. Machine learning methods have shown wide applicability in bioinformatics and proved invaluable for extracting useful information from the secondary structures of genomes by perfecting their techniques based on the mathematical archetype as opposed to the standard biological model of analysis. However, there are still several problems that motivate new approaches, mainly involving problems of feature extraction in extremely small structures such as snoRNAs. Considering this, the focus of the work is to study and analyze the feature extraction algorithms based on mathematical models such as numerical mapping, Fourier transformation, entropy and complex networks, having as a case study the two main classes of snoRNAs: \textit {C/D box} and \textit{H/ACA box}. Concisely, we define the case study in a pipeline divided into cycles based on the bases of \textit{machine learning} that will guide the research solidifying the comparison between the adopted mathematical paradigm and casual biological methods.

\begin{keywords}
non-coding RNAs, snoRNAs, Machine Learning, Mathematical Feature Extraction
\end{keywords}